\section{TAD \tadNombre{Coordenada}}

La operación distancia se incluyó dentro del TAD coordenada porque consideramos que es una característica intrínseca de las coordenadas, no del mapa.

\begin{tad}{\tadNombre{Coordenada}}
\tadGeneros{coordenada}
\tadExporta{generadores, observadores, distancia}
\tadUsa{\tadNombre{nat}, \tadNombre{coordenada}}

\tadIgualdadObservacional{n}{m}{coordenada}{$lat(n) \igobs lat(m) \wedge$\\ $long(n) \igobs long(m)$}


\tadAlinearFunciones{distancia}{coordenada, coordenada}
\tadObservadores
\tadOperacion{lat}{coordenada}{nat}{}
\tadOperacion{long}{coordenada}{nat}{}

\tadGeneradores
\tadOperacion{crear}{ nat/lat, nat/long }{nat}{}

\tadOtrasOperaciones
\tadOperacion{distancia}{coordenada, coordenada}{nat}{}

\tadAxiomas[\paratodo{nat}{n, m}, \paratodo{coordenada}{c1, c2}]
\tadAlinearAxiomas{long(crear(n, m))}
\tadAxioma{lat(crear(n, m))}{n}
\tadAxioma{long(crear(n, m))}{m}
\tadAxioma{distancia(c1, c2)}{$\sqrt{ (x(c1)-x(c2))^2 + (y(c1)-y(c2))^2}$}

\end{tad}
