\documentclass[10pt, a4paper]{article}
\usepackage[paper=a4paper, left=1.5cm, right=1.5cm, bottom=1.5cm, top=3.5cm]{geometry}
\usepackage[utf8]{inputenc}
\usepackage[T1]{fontenc}
\usepackage[spanish]{babel}
\usepackage{indentfirst}
\usepackage{fancyhdr}
\usepackage{latexsym}
\usepackage{lastpage}
\usepackage{aed2-symb,aed2-itef,aed2-tad}
\usepackage[colorlinks=true, linkcolor=blue]{hyperref}
\usepackage{calc}

\newcommand{\f}[1]{\text{#1}}
\renewcommand{\paratodo}[2]{$\forall~#2$: #1}

\sloppy

\hypersetup{%
 % Para que el PDF se abra a página completa.
 pdfstartview= {FitH \hypercalcbp{\paperheight-\topmargin-1in-\headheight}},
 pdfauthor={Cátedra de Algoritmos y Estructuras de Datos II - DC - UBA},
 pdfkeywords={TADs básicos},
 pdfsubject={Tipos abstractos de datos básicos}
}

\parskip=5pt % 10pt es el tamaño de fuente

% Pongo en 0 la distancia extra entre ítemes.
\let\olditemize\itemize
\def\itemize{\olditemize\itemsep=0pt}

% Acomodo fancyhdr.
\pagestyle{fancy}
\thispagestyle{fancy}
\addtolength{\headheight}{1pt}
\lhead{Algoritmos y Estructuras de Datos II}
\rhead{$1^{\mathrm{er}}$ cuatrimestre de 2012}
\cfoot{\thepage /\pageref{LastPage}}
\renewcommand{\footrulewidth}{0.4pt}

\author{Algoritmos y Estructuras de Datos II, DC, UBA.}
\date{}
\title{Tipos abstractos de datos básicos}

\begin{document}

%Pagina de titulo e indice
\thispagestyle{empty}

\maketitle
\tableofcontents

\newpage

%TADS
\section{TAD \tadNombre{Jugador}}

\begin{tad}{\tadNombre{Jugador} \textbf{as} \tadNombre{nat}}
\end{tad}
\section{TAD \tadNombre{Pokemon}}

\begin{tad}{\tadNombre{Pokemon}}
\tadGeneros{pokemon}
\tadExporta{generadores, observadores básicos}

\tadIgualdadObservacionalSimple{$(\forall p_1,p_2:Pokemon)$ $(p_1\igobs p_2)$ $\Leftrightarrow$ $(tipo(p_1)\igobs tipo(p_2) \wedge id(p_1)\igobs id(p_2)$}



\tadObservadores
\tadOperacion{tipo}{pokemon}{string}{}
\tadOperacion{id}{pokemon}{nat}{}


\tadGeneradores
\tadOperacion{crear}{nat, string}{pokemon}{}


\tadAxiomas[\paratodo{nat}{id}, \paratodo{string}{tipo}]
\tadAlinearAxiomas{tipo(crear($id$, $tipo$))}
\tadAxioma{tipo(crear($id$, $tipo$))}{$tipo$}
\tadAxioma{id(crear($id$, $tipo$))}{$id$}

\end{tad}

\section{TAD \tadNombre{Coordenada}}

\begin{tad}{\tadNombre{Coordenada}}
\tadGeneros{coordenada}
\tadExporta{generadores, observadores, distancia}
\tadUsa{\tadNombre{nat}, \tadNombre{coordenada}}

\tadIgualdadObservacional{n}{m}{coordenada}{$lat(n) \igobs lat(m) \wedge$\\ $long(n) \igobs long(m)$}


\tadAlinearFunciones{distancia}{coordenada, coordenada}
\tadObservadores
\tadOperacion{lat}{coordenada}{nat}{}
\tadOperacion{long}{coordenada}{nat}{}

\tadGeneradores
\tadOperacion{crear}{ nat/lat, nat/long }{nat}{$lat > 0 \wedge long > 0$}

\tadOtrasOperaciones
\tadOperacion{distancia}{coordenada, coordenada}{nat}{}

\tadAxiomas[\paratodo{nat}{n, m}, \paratodo{coordenada}{c1, c2}]
\tadAlinearAxiomas{long(crear(n, m))}
\tadAxioma{lat(crear(n, m))}{n}
\tadAxioma{long(crear(n, m))}{m}
\tadAxioma{distancia(c1, c2)}{$\sqrt{ (x(c1)-x(c2))^2 + (y(c1)-y(c2))^2}$}

\end{tad}


\end{document}
