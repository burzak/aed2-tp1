\section{TAD \tadNombre{Mapa}}

Pensamos el mapa como un conjunto de conjuntos de coordenadas. Dado que es importante para el juego qué coordenadas están conectadas entre sí, pero para el movimiento lo único que interesa es que haya un camino entre dos coordenadas y no por cuántas coordenadas hay que pasar para llegar de una a otra, decidimos separarlas en grupos (conjuntos) de coordenadas conectadas.

\begin{tad}{\tadNombre{Mapa}}
\tadGeneros{mapa}
\tadExporta{generadores, observadores, hayCamino, todasCoordenadas}
\tadUsa{\tadNombre{conj}, \tadNombre{bool}, \tadNombre{coordenada}}

\tadIgualdadObservacional{m_1}{m_2}{mapa}{$coordenadas(m_1) \igobs coordenadas(m_2)$}


\tadAlinearFunciones{todosCoordenadasAplanadas}{coordenada/c1, coordenada/c2, conj(conj(coordenada))/cc}
\tadObservadores
\tadOperacion{coordenadas}{mapa}{conj(conj(\newline coordenada))}{}

\tadGeneradores
\tadOperacion{crearMapa}{ conj(conj(coordenada))/c }{mapa}{$\forall conj1, conj2 \in c)((conj1 == conj2 \vee conj1 \cap conj2 == \emptyset)$}

\tadOtrasOperaciones
\tadOperacion{hayCamino}{coordenada/c1, coordenada/2, mapa/m}{bool}{$c1 \in todasCoordenadas(m) \wedge c2 \in todasCoordenadas(m)$}
\tadOperacion{hayCaminoRecursion}{coordenada/c1, coordenada/c2, conj(conj(coordenada))/cc}{bool}{$c1 \in todasCoordenadasAplanadas(cc) \wedge c2 \in todasCoordenadasAplanadas(cc)$}
\tadOperacion{todasCoordenadas}{mapa}{conj(coordenada)}{}
\tadOperacion{todosCoordenadasAplanadas}{conj(conj(coordenada))}{conj(coordenada)}{}

\tadAxiomas[\paratodo{mapa}{m}, \paratodo{coordenada}{c, c_1, c_2}, \paratodo{conj(conj(coordenada))}{cc}]
\tadAlinearAxiomas{$hayCaminoRecursion(c_1, c_2, cc)$}
\tadAxioma{coordenadas(crearMapa(c))}{c}

\tadAxioma{hayCamino($c_1$, $c_2$, m)}{$hayCaminoRecursion(c_1, c_2, coordenadas(m))$}

\tadAxioma{hayCaminoRecursion($c_1$, $c_2$, cc)}{\IF\ $\emptyset?(cc)$ THEN false ELSE {\IF\ $\{c_1, c_2\} \subseteq dameUno(cc)$ THEN true ELSE $hayCaminoRecursion(c_1, c_2, sinUno(cc))$ FI} FI}

\tadAxioma{todasCoordenadas(m)}{todasCoordenadasAplanadas(coordenadas(m))}

\tadAxioma{todasCoordenadasAplanadas(c)}{\IF\ $\emptyset?(c)$ THEN $\emptyset$ ELSE $dameUno(c) \bigcup todasCoordenadasAplanadas(sinUno(c))$ FI}

\end{tad}

