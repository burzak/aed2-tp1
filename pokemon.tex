\section{TAD \tadNombre{Pokemon}}

Decidimos implementar un número de identificación (id) para poder diferenciar entre pokemones del mismo tipo y facilitar otras estructuras (por ejemplo, al tener id se puede hacer un conjunto de pokemones con varios pikachus, sin necesidad de utilizar el TAD multiconjunto).

\begin{tad}{\tadNombre{Pokemon}}
\tadGeneros{pokemon}
\tadExporta{generadores, observadores básicos}
\tadUsa{\tadNombre{nat}, \tadNombre{string}}

\tadIgualdadObservacional{p_1}{p_2}{coordenada}{$tipo(p_1)\igobs tipo(p_2) \wedge$\\ $id(p_1)\igobs id(p_2)$}

\tadAlinearFunciones{crear}{nat, string}
\tadObservadores
\tadOperacion{tipo}{pokemon}{string}{}
\tadOperacion{id}{pokemon}{nat}{}


\tadGeneradores
\tadOperacion{crear}{nat, string}{pokemon}{}


\tadAxiomas[\paratodo{nat}{id}, \paratodo{string}{tipo}]
\tadAlinearAxiomas{tipo(crear($id$, $tipo$))}
\tadAxioma{tipo(crear($id$, $tipo$))}{$tipo$}
\tadAxioma{id(crear($id$, $tipo$))}{$id$}

\end{tad}
